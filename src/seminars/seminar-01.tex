% !TeX root = seminar-01.tex
% !TeX spellcheck = ru-RU
% !TeX encoding = UTF-8 Unicode
% !BIB program = biber
% LTeX: language=ru-RU
% LTeX: enablePickyRules=true
%
% Copyright (c) 2026 Михаил Михайлов
%
% This work is licensed under the Creative Commons Attribution-NonCommercial-ShareAlike 4.0
% International License. To view a copy of this license, visit:
% http://creativecommons.org/licenses/by-nc-sa/4.0/
%
% You are free to:
%   Share - copy and redistribute the material in any medium or format
%   Adapt - remix, transform, and build upon the material
%
% Under the following terms:
%   Attribution - You must give appropriate credit, provide a link to the license,
%                 and indicate if changes were made.
%   NonCommercial - You may not use the material for commercial purposes.
%   ShareAlike - If you remix, transform, or build upon the material, you must
%                distribute your contributions under the same license as the original.
%
% See the LICENSE file in the repository root for full license text.

\documentclass[seminar, recordsfile={../records.lua}]{practice}

\renewcommand{\SeminarName}{Немного кодирования}
\renewcommand{\SeminarDate}{16 января 2025}

\xsimsetup{
	solution/print = true
}
\begin{document}
\subsection*{Лемма Крафта}
Рассмотрим источник, который генерирует сообщения из алфавита 
\[
	\mathcal{A}=\set{a_{1}, a_{2}, \ldots, a_{n}}.
\]
Кодер преобразует символы в последовательности кодовых знаков из алфавита 
\[
	J = \set{0, 1,\ldots ,q-1}.
\]
Код --- это отображение 
\[
	f\colon  \mathcal{A} \longrightarrow J^{*},
\]
\begin{definition}
	Код называется \textbf{однозначно декодируемым}, если продолжение $f^*$ кода $f$ на $\mathcal{A}^{*} \to J^{*}$, определяемое правилом:
	\[
		f^*(u_1, \ldots, u_n) = f(u_1)\dots f(u_n)
	\]
	является инъективным отображением\footnote{Для двух слов $a$ и $b$ запись $ab$ обозначает их конкатенацию}.
\end{definition}
Далее под $l(a)$, где $a \in \mathcal{A}^*$ будем понимать длину кодового слова $|f^*(u)|$.

\begin{definition}
	Код называется \textbf{беспрефиксным}, если ни одно кодовое слово не является началом другого кодового слова, т.е. для любых \(a_1,a_2\in \mathcal{A}\) из условия \(a_2\neq a_2\) следует, что \(f(a_1)\) не является префиксом \(f(a_2)\) и наоборот.
\end{definition}
Беспрефиксные коды декодируются мгновенно: при чтении входного потока достаточно сразу определить конец текущего кодового слова без заглядывания вперёд.

\begin{exercise}
	Приведите пример кода, который однозначно декодируется, но не является беспрефиксным.
\end{exercise}
\begin{solution}
	Возьмём алфавит \(I=\{a,b, c\}\) и зададим код
	\[
		f(a)=1,\qquad f(b)=101, \qquad f(c) = 11011.
	\]
	Нетрудно проверить что такой код однозначно декодируется.
\end{solution}

\begin{lemma}[Лемма Крафта]
	Для алфавита источника $\mathcal{A}$ размера $n$ и алфавита кодировщика $J$ размера $q$, однозначно-декодируемый код $f \colon \mathcal{A} \to J$ с длинами слов $l_1, \ldots, l_n$ существует тогда и только тогда, когда 
	\[
		\sum_{i = 1}^{n} 2^{-l_i} \leqslant 1
	\]
\end{lemma}

В дальнейшем будем полагать \(q = 2\). Следующие четыре задачи доказывают необходимость условия Крафта.

\begin{exercise}
	Пусть \(f\colon \mathcal{A}\to J^{*}\) --- произвольный, \(\ell=\max_{a \in \mathcal{A}}l(a)\).
	Докажите, что для любого \(k\in\mathbb N\)
	\[
		\mleft(\sum_{a\in \mathcal{A}}2^{-l(a)}\mright)^{\!k}
		=
		\sum_{i=1}^{k\ell} \#\set{u \in \mathcal{A}^k| l(u)=i} \cdot 2^{-i}.
	\]
\end{exercise}
\begin{solution}
	Распишем левую часть:
	\[
		\mleft(\sum_{a\in \mathcal{A}}2^{-l(a)}\mright)^{\!k}
		=
		\mleft(\sum_{u_1\in \mathcal{A}}2^{-l(u_1)}\mright)
			\cdots
		\mleft(\sum_{u_k \in \mathcal{A}}2^{-l(u_k)}\mright)
	\]
	После раскрытия произведения получаем сумму по всем \(k\)-кратным упорядоченным наборам символов
	\[
		\sum_{(u_{1}, \dots ,u_{k}) \in \mathcal{A}^{k}} 2^{-l(u_1)} \cdots 2^{-l(u_k)}
		=
		\sum_{(u_{1}, \dots ,u_{k}) \in \mathcal{A}^{k}}
		2^{-\bigl(l(u_1)+\dots +l(u_{k})\bigr)}
		=
		\sum_{u \in \mathcal{A}^{k}} 2^{-l(u)}
	\]
	Группируя слагаемые по одинаковым значения длины \(i = l(u)\), получаем
	\[
		\sum_{u \in \mathcal{A}^{k}} 2^{-l(u)}
		=
		\sum_{i=1}^{k\ell}\#\set{u \in \mathcal{A}^k| l(u)=i} \cdot 2^{-i}.
	\]
	Что и требовалось доказать.
\end{solution}

\begin{exercise}
	Пусть теперь \(f\) --- однозначно‑декодируемый код (необязательно беспрефиксный). Докажите, что для любого \(i\in\mathbb N\) количество слов $u \in \mathcal{A}^*$, таких, что \(l(u) = i\), не превышает \(2^{i}\):
	\[
		\#\set{u \in \mathcal{A}^{*} | l(u) = i} \leq 2^{i}.
	\]
\end{exercise}
\begin{solution}
	Каждое кодовое слово длиной \(i\) является некоторой битовой строкой из множества \(J^{i} = \set{0,1}^{i}\). Поскольку в множестве \(J^{i}\) ровно \(2^{i}\) различных строк, количество различных кодовых слов любой длины не может превышать это число, иначе нарушится инъективность $f^*$.
\end{solution}

\begin{exercise}
	Пусть теперь \(f\) --- однозначно‑декодируемый код (необязательно беспрефиксный). Докажите, что для любого \(i\in\mathbb N\) количество слов $u \in \mathcal{A}^*$, таких, что \(l(u) = i\), не превышает \(2^{i}\):
	\[
		\mleft(\sum_{a\in \mathcal{A}}2^{-l(a)}\mright)^{\!k} \leq kl
	\]
\end{exercise}
\begin{solution}
	По предыдущим упражнениям:
	\[
		\mleft(\sum_{a\in \mathcal{A}}2^{-l(a)}\mright)^{\!k}
		=
		\sum_{i=1}^{k\ell} \underbrace{\#\set{u \in \mathcal{A}^k| l(u)=i}}_{\leq 2^i} \cdot 2^{-i}
		\leq
		\sum_{i=1}^{k\ell} 2^i \cdot 2^{-i}
		=
		k\ell
	\]
\end{solution}


\begin{exercise}
	Пусть \(f\) --- однозначно‑декодируемый код. Докажите, что
	\[
		\sum_{a \in \mathcal{A}} 2^{-l(a)} \leq 1.
	\]
\end{exercise}
\begin{solution}
	Возьмём произвольное \(k\in\mathbb N\) и применим предыдущее упражнение:
	\[
		\mleft(\sum_{a\in \mathcal{A}}2^{-l(a)}\mright)^{\!k} \leq k\ell
	\]
	Отсюда
	\[
		\sum_{a \in \mathcal{A}}2^{-l(a)} \leq (k\ell)^{1/k}.
	\]
	Поскольку \(\ell\) фиксировано, предел правой части при \(k\to\infty\) равен \(1\):
	\[
		\lim_{k\to\infty}(k\ell)^{1/k} = 1.
	\]
	Следовательно,
	\[
		\sum_{a \in \mathcal{A}}2^{-l(a)} \leq 1.
	\]
\end{solution}

\subsection*{Средняя длина и энтропия}
Будем теперь считать, что источник генерирует символ \(a_{i}\) с вероятностью \(p_{i}\), (все \(p_{i} > 0, \sum p_{i}=1\)).
\begin{definition}
	\textbf{Средняя длина} кода $f$ с длинами \(l_{i}=l(a_{i})\) определяется как
	\[
		L(C^f) = \sum_{i=1}^{n}p_{i}l_{i}.
	\]
\end{definition}
\begin{theorem}[Теорема Шенонна]
	Пусть $f$ --- однозначно-декодируемый код, имеющий наименьшую среднюю длину среди всех однозначно-декодируемых кодов. Тогда
	\[
		\mathrm{H}_{2}(p_{1},\dots ,p_{n}) \leq L(C^f) \leq \mathrm{H}_{2}(p_{1},\dots ,p_{n}) + 1
	\]
\end{theorem}

\begin{exercise}
	Докажите, что для любого однозначно‑декодируемого кода выполняется неравенство:
	\[
		L(C^f) \geq \mathrm{H}_{2}(p_{1},\dots ,p_n),
	\]
\end{exercise}
\begin{solution}
	Чтобы оценить нижнее значение, найдем минимум функции $L(C^f)$. Рассмотрим задачу минимизации средней длины при фиксированных вероятностях \(p_{i}\) и выполнении условия Крафта
	\begin{align*}
		&\sum_{i = 1}^{n} p_i l_i \to \min \\
		&\sum_{i = 1}^{n}2^{-l_{i}} \leq 1, \qquad l_{i} \in \N.
	\end{align*}
	Два замечания:
	\begin{itemize}[compact]
		\item Условие целочисленности $l_i$ можно ослабить --- это только ухудшит оценку минимума.
		\item Точка минимума находится на границе множества \(\sum_{i = 1}^{n}2^{-l_{i}} \leq 1\), т.е. когда 
		\[
			\sum_{i = 1}^{n}2^{-l_{i}} = 1.
		\]
		Поскольку, если бы точка минимума находилась внутри, то можно было бы уменьшить одно из $l_i$ на достаточно малое число --- это привело бы к тому что неравенство Крафта все еще сохранялось, но средняя длина уменьшилась. 
	\end{itemize}
	Итак, свелись к следующей задаче:
	\begin{align*}
		&\sum_{i = 1}^{n} p_i l_i \to \min \\
		&\sum_{i = 1}^{n}2^{-l_{i}} = 1, \qquad l_{i} \in \R.
	\end{align*}
	Запишем функцию Лагранжа:
	\[
		\mathcal{L}(l_{1},\dots ,l_{n},\lambda) = \sum_{i = 1}^{n}p_{i}l_{i}
		+ \lambda \cdot \left(\sum_{i = 1}^{n}2^{-l_{i}}-1\right).
	\]
	Найдём стационарные точки, дифференцируя по \(l_{i}\):
	\[
		\pd{\mathcal{L}}{l_i} = p_{i} - \lambda \cdot 2^{-l_{i}} \cdot \ln 2 = 0
		\Longrightarrow
		2^{-l_{i}} = \frac{p_{i}}{\lambda\ln 2}.
	\]
	Подставляя в условие Крафта:
	\[
		\sum_{i = 1}^{n} \frac{p_{i}}{\lambda\ln 2} = 1
		\Longrightarrow
		\lambda=\frac{1}{\ln 2}.
	\]
	Отсюда
	\[
		2^{-l_{i}} = p_{i}\quad\Longrightarrow\quad l_{i} = -\log_{2}p_{i}.
	\]
	Подставляя найденные \(l_{i}\) в целевую функцию, получаем
	\[
		L(C^f) \geq \mathrm{H}_{2}(p_{1}, \ldots , p_{n})
	\]
\end{solution}
\end{document}
