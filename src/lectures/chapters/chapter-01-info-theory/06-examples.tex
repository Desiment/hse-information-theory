
\subsection{Примеры вычисления энтропии}

\begin{example}[Энтропия бинарного события]
	Пусть $\mathcal{A} = \set{A, \overline{A}}$ --- некоторый опыт, имеющий два исхода. Обозначим за~$p = \P(A)$, $q = 1 - p = \P(\overline{A})$. Тогда
	\[
		\H(\mathcal{A}) = \H(p, q) = - p \log p - q \log q
	\]
	Функцию $\H(p,q)$ дальше для краткости будем обозначать $\H(p)$. 
\end{example}
Оказывается функция $\H(p)$ обладает парой полезных свойств.
\begin{proposition}
	Пусть $n, k \in \N$, $n \geq k$; Тогда для всех $m > 1$:
	\[
		\log_m\binom{n}{k} \leq n \cdot \H_m(\frac{k}{n}) 
	\]
\end{proposition}
\begin{proof}
	Пусть $p = \frac{k}{n}$. Тогда:
	\[
		1  = (p + (1-p))^n \geq \binom{n}{k}p^k(1-p)^{n-k} 
	\]
	После взятия логарифма от обоих частей получаем:
	\[
		0 \geq \log_{m}\binom{n}{k} - k \log_m p - (n-k)\log_m(1-p) = \log_m\binom{n}{k} - n \H_m(p)
	\]
	Откуда следует желанное.	
\end{proof}
Мы можем воспользоваться тем же приемом, чтобы получить еще более сильную оценку:
\begin{proposition}
	Пусть $n \in \N$. Тогда для всех $k \leq \floor{\frac{n}{2}}$ и всех $m > 1$:
	\[
		\sum_{i = 0}^{k}\binom{n}{i} \leq m^{n\H_m(\frac{k}{n})}
	\]
\end{proposition}
\begin{proof}
	Пусть $p = \frac{k}{n}$. Тогда:
	\[
		1  = (p + (1-p))^n \geq \sum_{i = 0}^{k}\binom{n}{i}p^i(1-p)^{n-i} \geq p^k(1-p)^{n-k}\sum_{i=0}^{k}\binom{n}{i}
	\]
	Последнее равенство выполнено в сиул того что $p \leq \frac{1}{2}$. Дальше логарифмируя и преобразовывая, получаем:
	\[
		n\H_m(\frac{k}{n}) \geq \log_{m}(\sum_{i = 0}^{k}\binom{n}{i})
	\]
	Откуда следует желанное.	
\end{proof}
\begin{corollary}
	Пусть $n \in \N$. Тогда для всех $k \geq \floor{\frac{n}{2}}$ и всех $m > 1$:
	\[
		\sum_{i = k}^{n}\binom{n}{i} \leq m^{n\H_m(\frac{k}{n})}
	\]
\end{corollary}
\begin{theorem}[О концентрации биномиального распределения]
	
\end{theorem}
\begin{proof}
	Положим $\alpha = \max(p,1-p)$. Тогда:
	\begin{multline*}
		\P(S_n \in [l_n; r_n]) = \sum_{k = l_n}^{r_n}\P(S_n = k) =\\ 
		=1 - \sum_{k = 0}^{l_n}\P(S_n = k) - \sum_{k = r_n}^{n}\P(S_n = k) 
		=1 - \sum_{k = 0}^{l_n}\binom{n}{k}p^kq^{n-k} - \sum_{k = r_n}^{n}\binom{n}{k} p^kq^{n-k}\geq\\
		\geq 1 - \alpha^n\sum_{k = 0}^{l_n}\binom{n}{k} - \alpha^n\sum_{k = r_n}^{n}\binom{n}{k} \geq
		1 - \alpha^n\cdot n \cdot \H(\frac{l_n}{n}) - \alpha^n \cdot n \cdot \H(\frac{r_n}{n})
	\end{multline*}
	Рапишем теперь $H$ в ряд Тейлора в окрестности точки $\frac{1}{2}$ (отметим что поскольку $\H(x + \frac{1}{2})$ четная, нечетные производные равны 0):
	\[
		\H(\frac{1}{2}+x) = \H(\frac{1}{2}) + x\od_{t=\frac{1}{2}}{\H_e(t)}{t} + o(x^2) = 1 + o(x^2)
	\]
	Подставляя в оценку выше, получаем:
	\[
		\P(S_n \in [l_n; r_n]) 
		\geq 1 - n \alpha^n \mleft(\frac{n}{2} - l_n\mright) - n \alpha^n \mleft(r_n - \frac{n}{2}\mright) + o\mleft(\alpha^n \cdot \frac{l_n^2+r_n^2}{n}\mright) 
		= 1 - n \alpha^n \mleft(r_n - l_n\mright) + o\mleft(\alpha^n \cdot \frac{l_n^2+r_n^2}{n}\mright) 
	\]
\end{proof}
\begin{corollary}
	\[
		\sum_{k = 0}^{n} \P(S_n = k)f\mleft(\P(S_n = k)\mright) \sim \int_{-\infty}^{\infty} f\mleft(\frac{1}{\sqrt{2\pi \sigma^2}}\exp(-\frac{(x-\mu)^2}{2\sigma^2})\mright) \exp(-\frac{(x-\mu)^2}{2\sigma^2}) \frac{\dif x }{\sqrt{2\pi\sigma^2}}  
	\]
\end{corollary}
\begin{proof}
	Хотим воспользоваться локальным Муарвом-Лаплассом:
	\begin{multline*}
		\sum_{k = 0}^{n} \P(S_n = k)f\mleft(\P(S_n = k)\mright)
		\sim
		\sum_{k = l_n}^{r_n} \P(S_n = k)f\mleft(\P(S_n = k)\mright)
		\sim\\\sim 
		\sum_{k = l_n}^{r_n} f\mleft(\frac{1}{n\sqrt{2\pi \sigma^2}}\exp(-\frac{(k-n\mu)^2}{2n\sigma^2})\mright) \exp(-\frac{(k-n\mu)^2}{2n\sigma^2}) \frac{1}{n\sqrt{2\pi\sigma^2}}
		=\\=
		\sum_{k = l_n}^{r_n} f\mleft(\frac{1}{n\sqrt{2\pi \sigma^2}}\exp(-\frac{(\frac{k}{\sqrt{n}}-\mu)^2}{2\sigma^2})\mright) \exp(-\frac{(k-n\mu)^2}{2n\sigma^2}) \frac{1}{n\sqrt{2\pi\sigma^2}}
	\end{multline*}
\end{proof}
\begin{example}[Энтропия биномиального распределения]
	Пусть $\mathcal{S}_{n, p}$ --- опыт, состоящий в проведении $n$ испытаний Бернулли с вероятностью успеха $p$ и измерении числа успехов. $S_k \in \mathcal{S}_{n,p}$ --- событие, означающее что произошло $k$ успехов\footnote{И само \(\mathcal{S}_{n,p} = \bigsqcup_{k} S_k\)}. Тогда
	\begin{multline*}
		\H(\mathcal{S}) =
		- \sum_{k = 0}^{n} \binom{n}{k}p^{k}q^{n-k} \log(\binom{n}{k}p^{k}q^{n-k})
		=\\= 
		- \sum_{k = 0}^{n} \binom{n}{k}p^{k}q^{n-k} \log\binom{n}{k} 
			- p \log p \underbrace{\sum_{k = 0}^{n} k\binom{n}{k}p^{k-1}q^{n-k}}_{\od{}{p}(p+q)^n = n\cdot(p+q)^{n-1}=n} 
			- q \log q\underbrace{\sum_{k = 0}^{n} (n-k)\binom{n}{k}p^{k}q^{n-k-1}}_{\od{}{q}(p+q)^n = n\cdot(p+q)^{n-1}=n}
		=\\=
		n \H(p,q) - \sum_{k = 0}^{n} \binom{n}{k}p^{k}q^{n-k} \log\binom{n}{k}  =
		n \H(p,q) - \sum_{k = 0}^{n} \binom{n}{k}p^{k}q^{n-k} \log\binom{n}{k} 
	\end{multline*}

	Далее
\end{example}


